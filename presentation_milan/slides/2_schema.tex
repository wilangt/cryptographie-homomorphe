\section{Le schéma de \textsc{J.H. Cheon} et \textsc{D. Sthelé}}

% essayer de faire une diapositive de début de partie 

\begin{frame}{Le schéma de \textsc{Jung Hee Cheon} et \textsc{Damien Stehlé} : Fully Homomophic Encryption over the Integers Revisited}
\framesubtitle{2016}
%Petit point sur l'article 
\end{frame}

\subsection{Partie théorique}



\begin{frame}{Le schéma de \textsc{Jung Hee Cheon} et \textsc{Damien Stehlé}}
    
    \begin{tikzpicture}
    % draw horizontal line   
    \draw (0,0) -- (10,0);
    
    % draw vertical lines
    \foreach \x in {0,2,5.5,4,6,8,8.5}
      \draw (\x cm,3pt) -- (\x cm,-3pt);

    % draw nodes
    \draw (0,0) node[below=3pt] {$ 0 $} node[above=3pt] {$   $};
    \draw (2,0) node[below=3pt] {$ p $} node[above=3pt] {$   $};
    \draw (4,0) node[below=3pt] {$ 2p $} node[above=3pt] {$  $};
    \draw (6,0) node[below=3pt] {$ 3p $} node[above=3pt] {$  $};
    \draw (8,0) node[below=3pt] {$ 4p $} node[above=3pt] {$  $};
    \draw (8.5,0) node[below=3pt] {$  $} node[above=3pt] {$4p+r$};
    \draw (5.5,0) node[below=3pt] {$  $} node[above=3pt] {$3p-q$};
  \end{tikzpicture}
  Avec $0 < q \ll p $ et $0<r \ll p$
    
    \color{white}
    t
    \newline
    tttt
    \color{black}
    
    \begin{alertblock}{Chiffrement sur les entiers}
    Problème du \textbf{PGCD approché}
    \end{alertblock}
    
\end{frame}

\begin{frame}{Le schéma de \textsc{Jung Hee Cheon} et \textsc{Damien Stehlé}}
	\begin{alertblock}{Génération des clefs}
		clef \textbf{privée} : $ sk = p $ \newline
		clef \textbf{publique} : $ pk = x_1, \ldots , x_\tau / \forall i \in [ 1, \tau ], x_i \leftarrow p q_i + r_i $
		avec $ x_0 $ le plus grand et $ \frac{x_1}{2} $ impair. %ajouter partie entiere
	\end{alertblock}
	
	\begin{alertblock}{Paramètres principaux}
	    $ \lambda $ : Paramètre de sécurité \\
	    $\eta$ : Nombre de bits de $p$ \\
	    $\gamma$ : Nombre de bits des $x_i$ \\
	    $\rho$ : Nombre de bits du bruit initial \\
	    $\tau$ : Nombre d'éléments dans la clef publique

	\end{alertblock}

\end{frame}

\begin{frame}
\frametitle{Fonctions de base}
	\begin{alertblock}{Chiffrer}
		Soit $S \subset \{1,2,\cdots,\tau\}$ \\
		$$c =  \left[ \sum_{i \in S} x_i + \left\lfloor \frac{x_1}{2} \right\rceil m \right]_{x_0}$$
	\end{alertblock}
	\begin{alertblock}{Déchiffrer}
		$$m= \left[ \left\lfloor \frac{2 c}{p} \right\rceil \right]_2$$
	\end{alertblock}
\end{frame}

\begin{frame}
\frametitle{Fonctions de base}
	\begin{alertblock}{Addition}
		$$c_{add} = [ c_1 + c_2 ]_{x_0}$$
	\end{alertblock}
		\begin{alertblock}{Multiplication}
		Procédé complexe
	\end{alertblock}
\end{frame}