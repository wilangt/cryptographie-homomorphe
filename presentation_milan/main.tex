
\RequirePackage{currfile} 

\documentclass{beamer}

\definecolor{rougeb}{rgb}{0.75,0.01,0}

%\setbeamercolor{subsection in toc}{fg=red} %change la couleur de la ToC
\setbeamerfont{section number projected}{family=\rmfamily,series=\bfseries,size=\normalsize}
\setbeamercolor{section number projected}{bg=rougeb,fg=white}
\setbeamerfont{subsection number projected}{family=\rmfamily,series=\bfseries,size=\normalsize}
\setbeamercolor{subsection number projected}{bg=rougeb,fg=white}

\input{preambule/special_beamer.tex}

% Les autres packages utiles  notamment pour le français, les accents ou Python
\usepackage{natbib}         % Pour la bibliographie
\usepackage{url}            % Pour citer les adresses web
\usepackage[T1]{fontenc}    % Encodage des accents
\usepackage[utf8]{inputenc} % Lui aussi
\usepackage[frenchb]{babel} % Pour la traduction française
\usepackage{numprint}       % Histoire que les chiffres soient bien

\usepackage{amsmath}        % La base pour les maths
\usepackage{mathrsfs}       % Quelques symboles supplémentaires
\usepackage{amssymb}        % encore des symboles.
\usepackage{amsfonts}       % Des fontes, eg pour \mathbb.

\usepackage{cancel}

%\usepackage[svgnames]{xcolor} % De la couleur

%%% Si jamais vous voulez changer de police: décommentez les trois 
%\usepackage{tgpagella}
%\usepackage{tgadventor}
%\usepackage{inconsolata}

%%% Pour L'utilisation de Python
\usepackage{minted}
\usemintedstyle{friendly}

\usepackage{graphicx} % inclusion des graphiques
\usepackage{wrapfig}  % Dessins dans le texte.

\usepackage{listings}

\definecolor{codegreen}{rgb}{0,0.6,0}
\definecolor{codegray}{rgb}{0.5,0.5,0.5}
\definecolor{codepurple}{rgb}{0.58,0,0.82}
\definecolor{backcolour}{rgb}{0.95,0.95,0.92}
 
\lstdefinestyle{mystyle}{
    %backgroundcolor=\color{backcolour},   
    commentstyle=\color{codegreen},
    keywordstyle=\color{magenta},
    numberstyle=\tiny\color{codegray},
    stringstyle=\color{codepurple},
    basicstyle=\footnotesize,
    breakatwhitespace=false,         
    breaklines=true,                 
    captionpos=b,                    
    keepspaces=true,                 
    numbers=left,                    
    numbersep=5pt,                  
    showspaces=false,                
    showstringspaces=false,
    showtabs=false,                  
    tabsize=2
}

\lstset{style=mystyle}

\usepackage{tikz}     % Un package pour les dessins (utilisé pour l'environnement {code})
\usepackage[demo]{tikzpeople}
\usepackage[framemethod=TikZ]{mdframed}
\usepackage{chronosys}

\usepackage{wasysym}

% Les macros et raccourcis personnels
\input{preambule/macros.tex}

% On définit le titre et l'auteur du document

\title[Cryptographie homomorphe]{Étude et implémentation d'un schéma de chiffrement homomorphe}
\author{Milan \textsc{Gonzalez-Thauvin}}
\institute[ ]{Thème : Transport}
\date{TIPE session 2019}

% On démarre le document proprement dit
\begin{document}

% Rien d'autre à faire qu'afficher le titre
\begin{frame}
\titlepage 
\end{frame}


% La table des matières utilise ce que vous donnez aux commandes \section et 
% \subsection tout au long de la présentation.
\begin{frame}
\frametitle{Plan de l'exposé} 
\tableofcontents 
\end{frame}

% Introduction
\section{Introduction}

\subsection{Cryptographie}

\begin{frame}
\frametitle{La cryptographie}
\framesubtitle{Introduction et histoire}
    
\end{frame}

\subsection{Cryptographie homomorphe}

\begin{frame}
\frametitle{La cryptographie homomorphe}
    
\end{frame}

\begin{frame}{Cryptographie Homomorphe}
    
\end{frame}

% Schéma
\section{Le schéma de Damien sthelé}

% essayer de faire une diapositive de début de partie 

\subsection{Partie théorique}

\begin{frame}{Principe mathématique}
	\begin{block}{Le schéma de Damien Sthélé : Cryptographie sur les entiers}
		clef privée : $ sk = p $ 
		clef publique : $ pk = x_1, \cdots , x_r / \forall i \in [ 1,r ], x_i \leftarrow p q_i + r_i $
		avec $ x_0 $ le plus grand et $ \frac{x_1}{2} $ impair. %ajouter partie entiere
	\end{block}

\end{frame}

\begin{frame}
\frametitle{Principe}
\framesubtitle{La théorie}
\end{frame}


\begin{frame}
\frametitle{Les fonctions}
\framesubtitle{add, mul}
\end{frame}

% Implémentation
\subsection{En pratique}

\begin{frame}{Implémentation (\textsc{Python 3})}
    \begin{alertblock}{État de l'implémentation}
    \begin{itemize}
    \item Création des clefs: \checked
    \item Chiffrement : \checked
    \item Déchiffrement : \checked
    \item Somme : \checked
    \item Produit : \checked
    \item Bootstrap : Échec
\end{itemize}
    \end{alertblock}{}
    
    \begin{alertblock}{Paramètres de base}
    Ceux proposés par l'article
    \end{alertblock}{}
\end{frame}


\begin{frame}
\frametitle{Constatation des performances}
\framesubtitle{Génération des clefs}
\begin{center}
\includegraphics[scale=0.46]{images/generation_clefs.png} 
\end{center}
\end{frame}

\begin{frame}
\frametitle{Constatation des performances}
\framesubtitle{Génération des chiffrés}
\begin{center}
\includegraphics[scale=0.46]{images/generation_chiffre2.png} 
\end{center}
\end{frame}

\begin{frame}
\frametitle{Constatation des performances}
\framesubtitle{Somme}
\begin{center}
\includegraphics[scale=0.46]{images/somme.png} 
\end{center}
\end{frame}

\begin{frame}
\frametitle{Constatation des performances}
\framesubtitle{Produit}
\begin{center}
\includegraphics[scale=0.46]{images/produit.png} 
\end{center}
\end{frame}

\begin{frame}{Constatation des performances}
\framesubtitle{Nombre d'opérations}
    \begin{alertblock}{Nombre d'opérations}
    \textbf{Dérisoire} $\to$ 4 ou 5 multiplications bit à bit avant erreur
    \end{alertblock}{}
\end{frame}{}
% d'ou ENJEU


\begin{frame}
\frametitle{Le bootstrap}
\begin{alertblock}{Échec}
En cause :
\begin{itemize}
    \item \textbf{Schéma trop théorique} et peu adapté à une implémentation
    \item \textbf{Évaluation homomorphique} des fonctions de chiffrement et déchiffrement \textbf{trop gourmande en calculs} et donc incompatible avec des paramètres pertinents pour un ordinateur
\end{itemize}{}
    \end{alertblock}
\end{frame}

% Améliorations
\section{Améliorations}

\subsection{Surcouche}

\begin{frame}
\frametitle{Surcouche}
\end{frame}

\subsection{Performances}

\begin{frame}
\frametitle{Paramètres}
\end{frame}

\subsection{Sécurité}

\begin{frame}
\frametitle{Génération des clefs}
\end{frame}

% Application

\section{Application}

\subsection{Contexte}

\begin{frame}
\frametitle{Contexte}
\end{frame}

\begin{frame}
\frametitle{Mise en forme}
\end{frame}

\subsection{Résultats}

\begin{frame} %Utile ?
\frametitle{Fonctionne ?}
\end{frame}

\begin{frame}
\frametitle{Performance}
\end{frame}

\begin{frame}
\frametitle{Sécurité}
\end{frame}

\section{Conclusion}

\begin{frame}
\frametitle{Conclusion}
\end{frame}

\end{document}
