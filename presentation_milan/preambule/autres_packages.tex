\usepackage{natbib}         % Pour la bibliographie
\usepackage{url}            % Pour citer les adresses web
\usepackage[T1]{fontenc}    % Encodage des accents
\usepackage[utf8]{inputenc} % Lui aussi
\usepackage[frenchb]{babel} % Pour la traduction française
\usepackage{numprint}       % Histoire que les chiffres soient bien

\usepackage{amsmath}        % La base pour les maths
\usepackage{mathrsfs}       % Quelques symboles supplémentaires
\usepackage{amssymb}        % encore des symboles.
\usepackage{amsfonts}       % Des fontes, eg pour \mathbb.

\usepackage{cancel}

%\usepackage[svgnames]{xcolor} % De la couleur

%%% Si jamais vous voulez changer de police: décommentez les trois 
%\usepackage{tgpagella}
%\usepackage{tgadventor}
%\usepackage{inconsolata}

%%% Pour L'utilisation de Python
\usepackage{minted}
\usemintedstyle{friendly}

\usepackage{graphicx} % inclusion des graphiques
\usepackage{wrapfig}  % Dessins dans le texte.

\usepackage{tikz}     % Un package pour les dessins (utilisé pour l'environnement {code})
\usepackage[framemethod=TikZ]{mdframed}
